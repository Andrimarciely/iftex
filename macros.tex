%% Macros de dados do documento
\autor{Autor do projeto}
\titulo{Título do projeto}
\instituicao{Instituto federal do espírito santo}
\curso{Curso de Engenharia Elétrica}
\local{Vitória}
\data{2019}


\tipotrabalho{Trabalho de Graduação}
\preambulo{Trabalho de Conclusão de Curso apresentado à Coordenadoria do Curso de Engenharia Elétrica do
Instituto Federal de Educação, Ciência e Tecnologia do Espírito Santo, como requisito parcial para a obtenção do título de Engenheiro Eletricista.}

\orientador[Orientador][Prof. Dr.]{Beltrano}[de Tal]{Instituto Federal do Espírito Santo}
\coorientador[Coorientadora][Profa. Dra.]{Beltrana}[de Tal]{Instituto Federal do Espírito Santo}

\examinadori{Profa. Dra. Fulana de Tal}{Instituto Federal do Espírito Santo}{Examinadora}
\examinadorii{Prof. Dr. Cicrano de Tal}{Instituto Federal do Espírito Santo}{Examinador}

\approvaldate{01}{Agosto}{2019}

% Palavras-chaves
\palavraschave{Palavra Chave 1}[Palavra Chave 2][Palavra Chave 3][Palavra Chave 4][Palavra Chave 5]
\keywords{keyword 1}[keyword 2][keyword 3][keyword 4][keyword 5]

% Ficha catalográfica
\fichabiblioteca{
    Dados Internacionais de Catalogação-na-Publicação (CIP) \\
    (Biblioteca Nilo Peçanha do Instituto Federal do Espírito Santo)
}
\fichaautor{Elaborada por XXXXXXXXXXXXXXXXXXXXX – CRB-X/ES - XXX}
\fichatags{
    1. Redes neurais (Computação).  2. Expressão facial – Processamento de dados. 3. Sistemas de reconhecimento de padrões. 4. Processamento de imagens – Técnicas digitais. 5. Percepção de padrões. 6. Engenharia Elétrica. I. XXXXXXXX, XXXXXXXXXX XXXXXXXXXX.  II. Instituto Federal do Espírito Santo. III. Título.
}

\cutter{X999y}
\cdd{000.00}
\selectlanguage{brazil}

% Arial (para usar times-new-roman, comente as três linhas abaixo)
\usepackage{helvet}
\renewcommand{\familydefault}{\sfdefault}
\renewcommand{\ABNTEXchapterfont}{\bfseries}
% Comandos para revisar o texto
\usepackage{easyReview} 


%% Elementos opcionais
\editardedicatoria{A dedicatória é um elemento opcional. Contém o oferecimento do trabalho adeterminada pessoa ou a pessoas (ASSOCIAÇÃO BRASILEIRA DE NORMAS TÉCNICAS, 2011b).}
\editaragradecimentos{A um elemento opcional. Localiza-se após a folha de aprovação e deve ser dirigido àqueles que realmente contribuíram, de maneira relevante,para a elaboração do trabalho. Deve-se utilizar uma linguagem simples (ASSOCIAÇÃO BRASILEIRA DE NORMAS TÉCNICAS, 2011b).}
\editarepigrafe{\textbf{Epigrafe:} É um elemento opcional. É uma citação relacionada ao assunto do trabalho desenvolvido, seguida da indicação de autoria (ASSOCIAÇÃO BRASILEIRA DE NORMAS TÉCNICAS, 2011b).   Deve-se seguir as regras do uso da citação NBR 10.520/2002.}


%% Listas [Siglas e Simbolos]
\editarlistasiglas{\begin{siglas}
    \item[Ifes] Instituto Federal do Espírito Santo
\end{siglas}}
\editarlistasimbolos{\begin{simbolos}
    \item[$\lambda$] Letra grega labda
\end{simbolos}}