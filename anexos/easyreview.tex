\chapter{{\normalfont EasyReview}}

\lstset{
		keywordstyle=\bfseries\color{purple!60!black},
		commentstyle=\itshape\color{green!40!black},
% 		identifierstyle=\itshape,
		stringstyle=\itshape,
		showstringspaces=false,
		escapeinside={\%*}{*)},
		columns=flexible,
		breaklines=true,
breakindent=0pt}

\renewcommand\lstlistingname{Example}
\newcommand{\keyword}[1]{\textbf{\color{purple!60!black}{#1}}}

% ------
\textbf{\uppercase {The alert command}}

% ------
Command intended to claim author's attention to a given part of the text. In the following, it is possible to see an example:

\begin{center}
\begin{minipage}[ht]{0.45\textwidth}
\begin{lstlisting}[language=tex]
A text without the alert command. %*\keyword{$\backslash$alert}*){A text with the alert command}.
\end{lstlisting}
\end{minipage}
\hspace{10pt}
\begin{minipage}[ht]{0.45\textwidth}
A text without the alert command. \alert{A text with the alert command}.
\end{minipage}
\end{center}

% ------
\textbf{\uppercase {The highlight command}}

% ------
Command intended to claim author's attention to a given part of the text in a different to the ``alert'' command. In the following, it is possible to see an example:

\begin{center}
\begin{minipage}[ht]{0.45\textwidth}
\begin{lstlisting}[language=tex]
A text without the highlight command. %*\keyword{$\backslash$highlight}*){A text with the highlight command}.
\end{lstlisting}
\end{minipage}
\hspace{10pt}
\begin{minipage}[ht]{0.45\textwidth}
A text without the highlight command. \highlight{A text with the highlight command}.
\end{minipage}
\end{center}

% ------
\textbf{\uppercase {The remove command}}

% ------
Command which an author suggest to remove a given part of the text. In the following, it is possible to see an example:

\begin{center}
\begin{minipage}[ht]{0.45\textwidth}
\begin{lstlisting}[language=tex]
This text is not to be removed. %*\keyword{$\backslash$remove}*){This text is to be removed}.
\end{lstlisting}
\end{minipage}
\hspace{10pt}
\begin{minipage}[ht]{0.45\textwidth}
This text is not to be removed. \remove{This text is to be removed}.
\end{minipage}
\end{center}

% ------
\textbf{\uppercase {The add command}}

% ------
Command which an author suggest to add new text in a given part of the text. In the following, it is possible to see an example:

\begin{center}
\begin{minipage}[ht]{0.45\textwidth}
\begin{lstlisting}[language=tex]
This text was already in the text. %*\keyword{$\backslash$add}*){This text is been added now}.
\end{lstlisting}
\end{minipage}
\hspace{10pt}
\begin{minipage}[ht]{0.45\textwidth}
This text was already in the text. \add{This text is been added now}.
\end{minipage}
\end{center}

% ------
\textbf{\uppercase {The replace/substitute command}}

% ------
Both commands are equivalent. It shall be used when an author suggest to replace a given part of the text for a newer one. In the following, it is possible to see an example:

\begin{center}
\begin{minipage}[ht]{0.45\textwidth}
\begin{lstlisting}[language=tex]
%*\keyword{$\backslash$replace}*){This part of the text needs to be replaced}{for this newer part of the text}.
\end{lstlisting}
\end{minipage}
\hspace{10pt}
\begin{minipage}[ht]{0.45\textwidth}
\replace{This part of the text needs to be replaced}{for this newer part of the text}.
\end{minipage}
\end{center}

% ------
\textbf{\uppercase {The commentreview command}}

% ------
Command intended to claim author's attention to a given part of the text, giving some comments to provide more information. In the following, it is possible to see an example:

\begin{center}
\begin{minipage}[ht]{0.45\textwidth}
\begin{lstlisting}[language=tex]
%*\keyword{$\backslash$commentreview}*){This text will receive a comment.}{This is the comment I have!}.
\end{lstlisting}
\end{minipage}
\hspace{10pt}
\begin{minipage}[ht]{0.45\textwidth}
\commentreview{This text will receive a comment.}{This is the comment I have!}.
\end{minipage}
\end{center}